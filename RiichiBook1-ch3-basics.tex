%~~~~~~~~~~~~~~~~~~~~~~~~~~~~~~~~~~~~~~~~~~~~~~~~~
% Riichi Book 1, Rozdział 3: Podstawy
%~~~~~~~~~~~~~~~~~~~~~~~~~~~~~~~~~~~~~~~~~~~~~~~~~
\chapter{Podstawy Riichi Mahjonga}\label{ch:basic}
\thispagestyle{empty}

\section{Strategie}

Mahjong to mieszanka szczęścia i~umiejętności.
Istnieje zbiór zasad strategicznych, które warto opanować szlifując sztukę kamieniarską.
Samo zdobywanie umiejętności nie jest ani niezbędne, ani wystarczające, by wygrać mecz.
Mając szczęście, mało wprawny gracz może łatwo pokonać mocniejszych przeciwników.
Wyniki gier, przynajmniej na krótszą metę, są bardziej oparte na szczęściu niż umiejętnościach, niemniej jednak stosowanie się do reguł strategicznych jest niezbędne, aby polepszyć nasze wyniki w~dłuższej perspektywie.\footnote{Interesującym zagadnieniem jest zdefiniowanie ,,krótszej mety''. To znaczy: ile gier należy rozegrać, aby odróżnić silnych graczy od tych mniej wprawnych? Badania pokazują, że potrzeba rozegrać około 100 meczy, aby niezawodnie określić poziom naszego Mahjonga. Turnieje EMA zazwyczaj składają się tylko z ośmiu gier, więc ich wygrywanie wymaga także odrobiny szczęścia.\index{european@EMA}}
Ponadto, rozumiejąc wspomniane zasady można czerpać większą radość z~gry.

\bigskip
Z~powodu probabilistycznej natury Mahjonga dobre posunięcia nie zawsze dają najlepsze efekty.
Najlepsze decyzje to takie, które \emph{statystycznie} dają najlepszy rezultat.
Dlatego też ocena naszych decyzji wymaga rozpatrzenia wszystkich opcji pod kątem \emph{prawdopodobieństwa} (czyli \emph{statystyki}).
Na przykład:
\vspace{5pt}
\begin{screen}
\CenteredLargeHand{m44p667s123678~z666*}
\vspace{-10pt}Który kamień należy odrzucić? \vspace{-5pt}
\end{screen}

%\bigskip
\pagebreak
\noindent
Ta ręka wejdzie w \emph{tenpai} po odrzuceniu \InlineTile{p6} lub \InlineTile{p7}.
Porównajmy obie możliwości:
\bi\itemsep.1pt
\i Odrzut \InlineTile{p6} \hspace{1pt} $\Rightarrow$ czekamy na \InlineTile{p5~8} (2 rodzaje -- 8 kamieni),
\i Odrzut \InlineTile{p7} \hspace{1pt} $\Rightarrow$ czekamy na \InlineTile{m4~p6} (2 rodzaje -- 4 kamienie).
\ei
Mimo, że obie opcje dają dwustronne czekanie, czekanie \InlineTile{p5~8} jest znacznie lepsze.
Przy czekaniu \InlineTile{m4~p6}, dwie sztuki każdego kamienia są już przez nas wykorzystane.

\bigskip
Może być tak, że po wybraniu czekania \InlineTile{p5~8} żaden kamień do wygranej nie przyjdzie, podczas gdy pojawi się kilka \InlineTile{m4} i~\InlineTile{p6}.
Takie rzeczy się zdarzają w każdej grze z elementem losowym.
Nie należy w takim wypadku kwestionować swojej decyzji.
Była prawidłowa, choć zabrakło szczęścia. Lepiej zachować spokój i~grać dalej.

\bigskip
Zanim rozważymy praktyczne metody maksymalizowania wydajności kamieni (\emph{tile efficiency}) w~następnym rozdziale, omówimy kilka podstawowych zasad tejże wydajności.
Wprowadzone zostaną podstawowe powszechnie używane pojęcia wraz z~ich oryginalnymi, japońskimi nazwami (napisanymi {\jap taką czcionką}).
Ich znajomość jest o~tyle ważna, że występują w wielu internetowych dyskusjach na temat mahjongowych strategii.

\section{Podstawowe pojęcia}

\subsection{Kamienie}

Kamienie w~Mahjongu można podzielić na dwie kategorie --- liczby oraz honory\footnote{Dla wielu terminów nie spotkałem się z~ich standardowymi polskimi tłumaczeniami --- mogą być zatem niedokładne, a także nieoczywiste dla osób trzecich. Niestety, osłuchanie się z~odpowiednikami japońskimi lub chociaż angielskimi jest ważne dla graczy nieposługujących się na co dzień tymi językami --- tłum.}.

\subsubsection{Liczby}

\begin{screen}
\centering
\begin{tabular}{l c c}
{Znaki} & {\jap Manzu} & \LargeHand{m123456789} \\ [\sep]
{Żetony} & {\jap Pinzu} & \LargeHand{p123456789} \\ [\sep]
{Bambusy} & {\jap Souzu} & \LargeHand{s123456789} \\
\end{tabular}
\end{screen}

\bigskip \noindent
Dalej, liczby można podzielić na {\bf proste} ({\jap tanyao hai}: kamienie od 2 do 8) oraz {\bf terminale} ({\jap yaochu hai}: 1 oraz 9).
To rozróżnienie jest istotne później, przy budowaniu układów ({\jap yaku}), a także przy liczeniu {\bf małych punktów} ({\jap fu}).
	\index{simple tiles ({\jap tanyao hai})} \index{tanyao@{\jap tanyao hai} (simple tiles)}
	\index{terminal tiles}
	\index{fu@{\jap fu} (minipoint)} \index{minipoint ({\jap fu})}

\bigskip
Ponadto, w~swoich zestawach (oraz oczywiście w Mahjongu komputerowym) możecie znaleźć czerwone piątki.
Przykładowo, jeden z~najpopularniejszych trybów na {\jap Tenhou} zawiera po jednej czerwonej piątce każdego rodzaju: \InlineTile{m0~p0~s0}.
Czerwone piątki zastępują zwykłe podczas gry: przykładowo, we wspomnianym trybie dla każdego rodzaju kamieni liczbowych będziemy mieć trzy zwykłe piątki oraz jedną czerwoną.
Każda czerwona piątka to {\jap dora} niezależnie od wskaźnika {\jap dora}.
Gdy wskaźnikiem jest czwórka danego rodzaju, czerwona piątka tego rodzaju staje się podwójnym kamieniem {\jap dora}. 

\vfill
\subsubsection{Honory}
\index{honor tiles ({\jap jihai})} \index{jihai@{\jap jihai} (honor tiles)}

\begin{screen}
\centering\scriptsize
\begin{tabular}{l c c c c}
Smoki & \LargeHand{z5} & \LargeHand{z6} & \LargeHand{z7}\\
& Biały ({\jap haku}) & Zielony ({\jap hatsu}) & Czerwony ({\jap chun})\\ \\
Wiatry & \LargeHand{z1} & \LargeHand{z2} & \LargeHand{z3} & \LargeHand{z4} \\
& Wschód ({\jap ton}) & Południe ({\jap nan}) & Zachód ({\jap sha}) & Północ ({\jap pei})\\
\end{tabular}
\end{screen}

	\index{value tiles ({\jap fanpai; yakuhai})}
	\index{fanpai@{\jap fanpai} ({\jap yakuhai}; value tiles)}
	\index{yakuhai@{\jap yakuhai} ({\jap fanpai}; value tiles)}
	\index{valueless wind@valueless wind ({\jap otakaze})}
	\index{otakaze@{\jap otakaze} (valueless wind)}
\noindent
Niektóre z honorów są {\bf wartościowe} ({\jap fanpai / yakuhai}): dostajemy jeden {\jap han} jeśli zbierzemy grupę trzech tego samego rodzaju.
Wszystkie smoki są wartościowe niezależnie od wiatru rundy oraz gracza.
Wartościowość wiatrów zależy od rundy oraz naszego miejsca przy stole.
Kamienie wschodu są wartościowe dla wszystkich podczas rundy wschodu, południa --- podczas rundy południa.
Ponadto każdy z graczy ma swój własny wiatr zależny od miejsca przy stole oraz numeru rozdania.
Przykładowo, podczas rundy wschodu kamienie zachodu są wartościowe tylko dla gracza zachodu, ale są bezwartościowym wiatrem ({\jap otakaze}) dla pozostałych graczy.

\vfill
\subsection{Grupy ({\jap mentsu})}
\index{run ({\jap shuntsu}; chow)}
\index{chow ({\jap shuntsu}; run)}
\index{shuntsu@{\jap shuntsu} (run; chow)}
\index{set ({\jap kotsu}; pung)}
\index{pung ({\jap kotsu}; set)}
\index{kotsu@{\jap kotsu} (set; pung)}
\index{mentsu@{\jap mentsu} (group)}
\index{group ({\jap mentsu})}
\index{kan@{\jap kan} (kong)} \index{kantsu@{\jap kantsu} (quad)}
\index{quad@quad ({\jap kantsu})}

Jednym z głównych celów Mahjonga jest wygranie ręki.\footnote{Innym ważnym celem jest nieodrzucenie kamienia, na którym wygrywa przeciwnik. Strategie defensywne będą omówione w rozdziale \ref{ch:defense}. Niemniej najważniejszym celem jest wygranie gry, a gra defensywna czy też wygranie pojedynczej ręki to jedynie środki do jego osiągnięcia. Więcej na ten temat pojawi się w rozdziale \ref{ch:grand}.}
Standardowo rękę wygrywa się poprzez złożenie czterech {\bf grup} ({\jap mentsu}) oraz jednej {\bf głowy} ({\jap atama}).\footnote{Istnieją trzy wyjątki od tej reguły:
{\jap chiitoitsu} (siedem par), {\jap kokushi musou} (trzynaście sierot) oraz {\jap nagashi mangan} (odrzuty terminali i honorów).}
Grupy można podzielić na dwa rodzaje --- {\bf sekwensy} oraz {\bf trójki}.\footnote{EMA nazywa je odpowiednio ,,chow'' oraz ,,pung''.}
\bi
\i Sekwens ({\jap shuntsu}) to trzy kolejne kamienie liczbowe, np. \InlineTile{m789}, \InlineTile{s345}.
\i Trójka ({\jap kotsu}) to trzy identyczne kamienie np. \InlineTile{m222} czy też \InlineTile{z666}.\footnote{Istnieje trzeci rodzaj grup: czwórki ({\jap kantsu}; ,,kong'' w regułach EMA), czyli zbiór czterech identycznych kamieni. Traktujemy je jako wariant trójek. Więcej w sekcji \ref{sec:kong}.}
\ei
 

\subsection{{\jap Tenpai} i {\jap $n$-shanten}}
	\index{ready ({\jap tenpai})} \index{tenpai@{\jap tenpai} (ready)}
	\index{1-away (1-{\jap shanten})}
	\index{2-away (2-{\jap shanten})}
	\index{3-away (3-{\jap shanten})}
	\index{ukeire@{\jap ukeire} (tile acceptance)} 
	\index{tile acceptance ({\jap ukeire})}
Kiedy czekamy na ostatni kamień do wygranej, to mówimy, że nasza ręka jest {\jap tenpai} (\emph{gotowa}).
Przykładowo, poniższa ręka jest {\jap tenpai}.
\bigskip
\begin{screen}
\CenteredLargeHand{p34588s23678z777}
\end{screen}
Rękę uzupełni jeden z dwóch kamieni: \InlineTile{s1} lub \InlineTile{s4}. Mówimy, że ręka {\bf czeka} na \InlineTile{s1~4}.

\bigskip

Gdy brakuje nam jednego kamienia do {\jap tenpai}, to nasza ręka jest 1-{\jap shanten}.
Przykładowo, poniższa ręka jest 1-{\jap shanten}.
\bigskip
\begin{screen}
\CenteredLargeHand{p35588s23678z777}
\end{screen}
Ręka staje się {\jap tenpai} po dobraniu któregokolwiek z \InlineTile{p458s14}.
Mówimy, że ręka {\bf akceptuje} \InlineTile{p458s14} (5 rodzajów -- 16 kamieni), gdyż każdy z nich sprawia, że ręka staje się {\jap tenpai}.

{\bf Akceptowalność} ({\jap ukeire}) to termin odnoszący się do rodzajów i liczby kamieni, które dana ręka akceptuje.
Gdyby nie brać innych czynników pod uwagę, ręka 1-{\jap shanten} akceptująca więcej kamieni jest lepsza od innej 1-{\jap shanten} z mniejszą akceptowalnością.

\bigskip
Bardziej ogólnie, ręka jest $n$-{\jap shanten}, gdy da się ją zrobić {\jap tenpai} po $n$ krokach.
Poniższa ręka jest 2-{\jap shanten}:
\bigskip
\begin{screen}
\CenteredLargeHand{p35588s23677z777}
\end{screen}
\vspace{5pt}
\noindent
Ta ręka akceptuje te same kamienie, które akceptowała ręka 1-{\jap shanten} powyżej (\InlineTile{p458s14}) wraz z dodatkowymi kamieniami \InlineTile{p3s235678}.\footnote{Dobranie \InlineTile{p3s236} powoduje, że jesteśmy 1-{\jap shanten} do {\jap chiitoitsu}.}
Dobranie któregokolwiek z tych kamieni powoduje, że ręka staje się 1-{\jap shanten}.

\bigskip
Ręka może być też 3-{\jap shanten}, 4-{\jap shanten}, 5-{\jap shanten} lub 6-{\jap shanten}.\footnote{6-{\jap shanten} jest wtedy, gdy w ręce nie ma żadnej pary --- wtedy potrzeba dobrać 6 kamieni, aby być {\jap tenpai} do {\jap chiitoitsu}.}
Jednakże w praktyce nie warto rozróżniać rąk 3-{\jap shanten} od 4-{\jap shanten} lub gorszych.
W związku z tym gracz powinien umieć rozróżniać cztery rodzaje rąk: {\jap tenpai}, 1-{\jap shanten}, 2-{\jap shanten} oraz pozostałe.

\subsubsection{Zmniejszanie akceptowalności}
Gdy $n$ maleje, a ręka jest coraz bliżej ułożenia, rodzaje oraz liczba akceptowanych kamieni będzie maleć.
Rozważmy to, jak postępuje rozwój ręki opisanej wyżej.
\bi
\i W 2-{\jap shanten}, akceptuje \InlineTile{p3458s12345678},
\i W 1-{\jap shanten}, akceptuje \InlineTile{p458s14},
\i W {\jap tenpai}, czeka na \InlineTile{s14}.
\ei
Akceptowalność jest najmniejsza, gdy ręka jest {\jap tenpai}.
Zauważmy, że jest ona także \emph{efektywnie} minimalna w 1-{\jap shanten}.
Wynika to stąd, że ręka {\jap tenpai} pozwala na wykorzystanie nie tylko kamieni dobranych, ale także odrzuconych przez innych.
Ręce $n$-{\jap shanten} muszą przy rozwoju polegać prawie wyłącznie na kamieniach, które samo dobieramy.\footnote{Meldunki ({\jap pon} / {\jap chii} / {\jap kan}) nie zawsze są możliwe. Przykładowo, podana wyżej ręka 2-{\jap shanten} akceptuje \InlineTile{s3} po dobraniu, ale nie można go zameldować ani {\jap pon}, ani {\jap chii}.}
A zatem, wybierając kamień do odrzucenia będąc 2-{\jap shanten}, powinniśmy starać się unikać wylądowania z ręką 1-{\jap shanten} i małą akceptowalnością.

\subsubsection{Rozwijanie ręki}

Aby wygrać rozdanie, potrzebujemy rozwijać naszą rękę redukując $n$ przy naszej $n$-{\jap shanten} ręce, dopóki nie będziemy {\jap tenpai}.
Kiedy ręka jest 2-{\jap shanten}, powinniśmy starać się zrobić z niej 1-{\jap shanten}.
Kiedy ręka jest 1-{\jap shanten}, powinniśmy starać się zrobić z niej {\jap tenpai}.
Przykładowo, rozważmy następującą sytuację:
\begin{screen}
\CenteredLargeHand{m5677p34s45579z666}
\vspace{-10pt}Co należy odrzucić? \vspace{-5pt}
\end{screen}
\noindent
Odrzucenie \InlineTile{s9} sprawia, że ręka staje się 2-{\jap shanten}, a odrzucenie \InlineTile{m7} lub \InlineTile{s4} sprawia, że ręka staje się 1-{\jap shanten}.
Należy zatem odrzucić \InlineTile{m7} lub \InlineTile{s4}.
Wycofanie się z ręki 1-{\jap shanten} do 2-{\jap shanten} ma sens tylko w wyjątkowych sytuacjach, gdy akceptacja w 1-{\jap shanten} staje się bardzo mała (mniej niż 2 rodzaje kamieni).
Powyższa ręka będzie akceptować \InlineTile{p25s8} (3 rodzaje -- 12 kamieni) w 1-{\jap shanten}.

\subsection{Niekompletne sekwensy ({\jap taatsu})} \label{sec:protorun}
	\index{protorun ({\jap taatsu})} \index{taatsu@{\jap taatsu} (protorun)}
	\index{waits!side wait} \index{waits!closed wait} \index{waits!edge wait}

Łatwiej jest skompletować sekwens niż trójkę. Każdy kamień występuje w czterech egzemplarzach, a do uzbierania trójki potrzebujemy aż trzech z nich.
Dlatego też zwykle będziemy skupiać się na składaniu sekwensów.

\bigskip
Parę kamieni, które można uzupełnić do sekwensu ({\bf niekompletny sekwens}) nazywamy {\jap taatsu}.
Mamy trzy rodzaje {\jap taatsu}:
\bigskip

{\begin{table}[h!]\centering \small \captionsetup{font=footnotesize}
\caption{Typy {\jap taatsu}} \label{tbl:proto}
\begin{tabular}{l l c c l} 
\toprule
Nazwa & Nazwa jap. & Przykład & Czekanie & {\jap Ukeire}\\ 
\midrule 
dwustronne & {\jap ryanmen} & {\LargeHand{s34}} & {\LargeHand{s25}} & 2 rodzaje -- 8 kamieni\\ [\sep]
środkowe & {\jap kanchan} & {\LargeHand{p24}} & {\LargeHand{p3}} & 1 rodzaj -- 4 kamienie\\ [\sep]
skrajne & {\jap penchan} & {\LargeHand{s89}} & {\LargeHand{s7}} & 1 rodzaj -- 4 kamienie\\ [\sep]
\bottomrule
\end{tabular}
\end{table}}

Jak widać powyżej, {\jap ryanmen} czeka na dwa razy więcej kamieni niż pozostałe {\jap taatsu}.
Jest on więc kluczem do posunięcia ręki naprzód.

\bigskip
\subsubsection*{Czekanie środkowe a skrajne}
Środkowe i skrajne czekanie nie różni się pod względem {\jap ukeire} --- przy obu czekamy na 1 rodzaj, 4 kamienie.
Środkowe czekanie ma jednak przewagę, ponieważ możemy ulepszyć je do dwustronnego.

\bigskip
Ulepszenie odbywa się w jednym kroku.
Przykładowo możemy, mając {\jap kanchan} \InlineTile{p13} i dobierając \InlineTile{p4}, zamienić je na na {\jap ryanmen} \InlineTile{p34} odrzucając \InlineTile{p1}.
\begin{center}
{\VCLargeHand{p13} ~~$\Rightarrow$~~ \VCLargeHand{p34}}\\
+ \VCInlineTile{p4}
\end{center}
Z drugiej strony skrajne czekanie potrzebuje aż dwóch kamieni, aby zamienić się na dwustronne: \InlineTile{p89} zamieni się w {\jap ryanmen} po dobraniu najpierw \InlineTile{p6}, a następnie \InlineTile{p5}.
\begin{center}
{\VCLargeHand{p89} ~~ $\Rightarrow$ ~~ \VCLargeHand{p68} ~~ $\Rightarrow$ ~~ \VCLargeHand{p56}}\\
+ \VCInlineTile{p6} ~~~~~~~~~~~~~~~ + \VCInlineTile{p5}
\end{center}

\bigskip

\color{MyRed}
\begin{itembox}[c]{Ranking {\jap taatsu}}
\normalcolor
\centering
czekanie dwustronne > czekanie środkowe > czekanie skrajne
\end{itembox}
\normalcolor

\bigskip

\subsubsection{Przydatność kamieni} \label{sec:versatility}

Niektóre kamienie się bardziej użyteczne od innych.
Kamienie numeryczne są najbardziej przydatne, gdyż kamienie specjalne nie mogą tworzyć sekwensów.
Ponadto, możemy określać przydatność kamieni w oparciu o rodzaje {\jap taatsu}, które mogą stworzyć.

\bigskip
Kamienie między 3 a 7 są najbardziej przydatne.
Wynika to z faktu, że każdy z nich tworzy {\jap taatsu} z czterema rodzajami kamieni.
Przykładowo, \InlineTile{p3} tworzy niekompletny sekwens z \InlineTile{p1}, \InlineTile{p2}, \InlineTile{p4} i \InlineTile{p5}.
Ponadto w dwóch przypadkach dostajemy {\jap ryanmen}.

\bigskip
2 i 8 są troszkę mniej elastyczne --- ,,parują się'' z trzema rodzajami kamieni.
Na przykład, \InlineTile{p2} tworzy {\jap taatsu} z \InlineTile{p1}, \InlineTile{p3} oraz \InlineTile{p4}.
Tylko jedna z trzech opcji kończy się dwustronnym czekaniem.

\bigskip
Terminale (1 i 9) są najmniej przydatne --- tworzą {\jap taatsu} z tylko 2 rodzajami kamieni. Na przykład \InlineTile{p1} da {\jap taatsu} tylko z \InlineTile{p2} oraz \InlineTile{p3}.
Żadna z tych opcji nie kończy się {\jap ryanmen}em.

\color{MyRed}
\begin{itembox}[c]{Ranking przydatności kamieni}
\centering\normalcolor
liczbowe 3--7 > liczbowe 2, 8 > liczbowe 1, 9 > honory
\end{itembox}
\normalcolor

Idąc tym tokiem rozumowania możemy też sklasyfikować przydatność {\jap kanchan}ów.
Przykładowo, {\jap kanchan} \InlineTile{p13} stworzy {\jap ryanmen} tylko z \InlineTile{p4}, podobnie jak \InlineTile{p24} stworzy {\jap ryanmen} tylko z \InlineTile{p5}.
Za to {\jap kanchan} \InlineTile{p35} zmieni się w {\jap ryanmen} po dobraniu któregokolwiek z kamieni \InlineTile{p2} lub \InlineTile{p6}.
Wyraźnie tutaj widać, że niekompletna grupka \InlineTile{p35} jest bardziej elastyczna od \InlineTile{p13} lub \InlineTile{p24}.

\color{MyRed}
\begin{itembox}[c]{Ranking elastyczności {\jap kanchan}ów}
\centering\normalcolor
35, 46, 57 > 13, 24, 68, 79
\end{itembox} 
\normalcolor
\bigskip

\subsection{Pary ({\jap toitsu})}\label{sec:closevers}
\index{pair ({\jap toitsu})} \index{toitsu@{\jap toitsu} (pair)}

Zbiór dwóch identycznych kamieni nazywamy {\bf parą} ({\jap toitsu}).
Pełnią one kilka ról: para może być końcową parą w układzie, kandydatem do trójki lub składnikiem siedmiu par ({\jap Chiitoitsu}).

\bigskip
\subsubsection{Składanie par}
Każdy układ --- włącznie z {\jap Kokushi Musou} i {\jap Chiitoitsu} --- musi mieć parę.
Złożenie pary jest bardzo proste, więc zwykle nie przejmujemy się nią zbytnio.
Na przykład:
\bigskip
\begin{screen}
\CenteredLargeHand{m789p234567s3457}
\end{screen}
Ta ręka nie ma pary, a czekanie nie jest zbyt dobre.
Układ uzupełnia jedynie \InlineTile{s7} (3 kamienie).
Jeśli dostaniemy któryś z \InlineTile{m69p124578s2356} (12 rodzajów -- 41 kamieni), czekanie znacznie się poprawi.
Przykładowo, po dobraniu \InlineTile{p5} ręka będzie wyglądała tak:
{\CenteredLargeHand{m789p2345567s345}}
\noindent Czekamy na \InlineTile{p258} (3 rodzaje -- 9 kamieni).
Gdy ręka nie ma pary, czekanie można zwykle łatwo ulepszyć.

\subsection{Pary i grupy} \label{sec:2pairs}
Kolejną ważną rolą par jest możliwość ich zamiany w grupkę.
Gdy mamy na ręce więcej niż jedną parę, możemy liczyć na pozyskanie grupki z jednej z nich, podczas gdy druga pozostanie naszą parą.
Innymi słowy, pary są najbardziej przydatne, gdy mamy dwie (nie więcej).
Aby lepiej to zobrazować, popatrzmy na poniższe przykłady rąk z jedną, dwiema i trzema parami.

%\newpage
\begin{screen}
\CenteredLargeHand{m135p56789s2267z4}
\end{screen}
\noindent
Ta ręka jest 2-{\jap shanten} i ma jedną parę, która nie jest dobrym kandydatem na grupkę z dwóch powodów.
Po pierwsze, jeśli dostaniemy kolejne \InlineTile{s2}, zyskamy grupkę, ale stracimy parę i nasza ręka pozostanie 2-{\jap shanten}.
Po drugie: szanse na kolejne \InlineTile{s2} są niewielkie --- zostały tylko dwie sztuki.

\bigskip
Co, jeśli ręka będzie miała dwie pary?
Załóżmy, że dobieramy \InlineTile{m1} i pozbywamy się \InlineTile{z4}.

\bigskip
\begin{screen}
\CenteredLargeHand{m1135p56789s2267}
\end{screen}
\noindent Ta ręka też jest 2-{\jap shanten}.
Każda z par może rozwinąć się do trójki.
Po dobraniu \InlineTile{m1} lub \InlineTile{s2} będziemy mieli 1-{\jap shanten}.

\bigskip
Dodatkowo, ręka z jedną parą czeka na dwie sztuki \InlineTile{s2}.
Ręka z dwiema parami czeka już na cztery kamienie.
Ogólnie rzecz biorąc, {\jap ukeire} zwiększa się o 2 kamienie przy każdej zebranej parze.

\bigskip
Co, jeśli ręka ma trzy pary? Przypuśćmy, że dobieramy \InlineTile{s6}:

\bigskip
\begin{screen}
\CenteredLargeHand{m1135p56789s2267~6}
\end{screen}
\noindent Jeśli zatrzymamy drugie \InlineTile{s6} i wyrzucimy \InlineTile{s7} lub \InlineTile{m5}, ręka będzie miała trzy pary.
Trzymanie trzech par na ręce jest niekorzystne.
Dla przypomnienia --- każda para zwiększa {\jap ukeire} o dwa kamienie, w tym przypadku dwie sztuki \InlineTile{s66}.
Nie odbywa się to jednak bez strat --- zatrzymując 3 pary, po wyrzuceniu \InlineTile{s7} pozbywamy się szans na \InlineTile{s5}.
Podobnie jest z odrzuceniem \InlineTile{m5} --- tracimy {\jap ukeire} w postaci \InlineTile{m4}.
W związku z tym najefektywniejszym odrzutem będzie \InlineTile{s6}.

\bigskip
Powyższe przykłady można przenieść na szerszy zakres sytuacji.
Jeśli chcemy złożyć zamknięty układ, powinniśmy unikać trzech par na ręce --- dają one najsłabszą kompozycję.
Dwie pary dają największe możliwości manewru.
Jeśli nasza ręka ma cztery pary, warto rozważyć {\jap Chiitoitsu}.

\bigskip

\color{MyRed}
\begin{itembox}[c]{Efektywność par: zamknięta ręka}
\centering\normalcolor
2 pary > 1 para, 4 pary > 3 pary
\end{itembox}\normalcolor

\bigskip

\subsubsection{Otwarta ręka}

Jest jeden kruczek dotyczący powyższej reguły.
Jeśli chcemy meldować {\jap pon}y, trzy pary są bardziej wskazane niż dwie.
Jest tak dlatego, że po meldunku nasza ręka będzie mieć dwie pary.
Przykładowo:
\begin{screen}
\CenteredLargeHand{m1135p45s244789z77~7}
\vspace{-15pt}\hfill{\footnotesize{\jap Dora}~~~~~~~}\\
\vspace{-2pt}Który kamień należy odrzucić?\vspace{-5pt}
\end{screen}
\noindent
Zdecydowanie chcemy zdobyć trójkę \InlineTile{z7}.
Mając to na uwadze, powinniśmy odrzucić \InlineTile{m5} i zachować trzy pary.
Po meldunku będziemy wybierać między \InlineTile{m3} a \InlineTile{s2}.
{\CenteredLargeHand{m11p45s244789~z77*7}}
{\CenteredLargeHand{m113p45s44789~z77*7}}
W obu przypadkach będziemy mieli dwie pary po meldunku.

\bigskip

\color{MyRed}
\begin{itembox}[c]{Efektywność par: otwarte ręce}
\centering\normalcolor
3 pary > 2 pary
\end{itembox}\normalcolor

\bigskip

\subsection{Perfect $n$-away}\label{sec:perfect}

\subsubsection{Perfect 1-away} 
	\index{1-away (1-{\jap shanten})!perfect 1-away}
When a 1-away hand has two side-wait protoruns and two pairs, it is called {\bf perfect 1-away}. 
\bigskip
\begin{itembox}[r]{Perfect 1-away}
\bp
\wan{2}\wan{3}\wan{3}\tong{1}\tong{2}\tong{3}\tong{6}\tong{6}\suo{6}\suo{7}\bai\bai\bai\\
\ep
\end{itembox}
The hand above is an example of perfect 1-away. It is called ``perfect'' because this hand can become ready either by calling {\jap chii}, calling {\jap pon}, or drawing a tile to complete a run or a set, and no matter how a hand becomes ready, you will \emph{always} have the option to choose side wait as the final wait.

\subsubsection{Perfect 2-away}
	\index{2-away (2-{\jap shanten})!perfect 2-away}
One step prior to achieving perfect 1-away, we may get a perfect 2-away hand. Perfect 2-away is made up with three side-wait protoruns and three pairs, as follows.
\bigskip
\begin{itembox}[r]{Perfect 2-away}
\bp
\wan{2}\wan{3}\wan{3}\tong{2}\tong{3}\tong{6}\tong{6}\suo{6}\suo{7}\suo{7}\bai\bai\bai\\
\ep
\end{itembox}
When a perfect 2-away hand becomes 1-away, it can always be perfect 1-away (unless you choose not to, for some reason). However, not all perfect 1-away hands evolve from a perfect 2-away hand. 

\newpage
\subsection{Putting things all together: an example}\label{sec:ptat}

Let's see some hand examples that illustrate how we can apply the tile efficiency logics we have learned so far. Consider the following 2-away hand. 
\begin{itembox}[r]{Advancing a hand 1}
\bp
\wan{5}\wan{5}\wan{7}\suo{2}\suo{4}\suo{8}\suo{8}\suo{9}\tong{2}\tong{3}\tong{7}\tong{8}\tong{9}~\suo{2}\\
\hfill\footnotesize{Draw~~~~~~~~~~~}
\ep
\vspace{-17pt}What would you discard? \vspace{-5pt}
\end{itembox}
\noindent
The hand now has three pairs, and we should avoid it. In order to reduce the number of pairs in this hand from three to two, our discard candidates should be {\LARGE\wan{5}}, {\LARGE\suo{2}}, or {\LARGE\suo{8}}. Which one should we choose? 

\bigskip
Recall that a closed-wait protorun of 57 is stronger than a closed-wait protorun of 24 or an edge-wait protorun of 89. Therefore, it is OK to cut down the {\LARGE\wan{5}\wan{5}\wan{7}} shape to {\LARGE\wan{5}\wan{7}} by discarding {\LARGE\wan{5}}. This is because {\LARGE\wan{5}\wan{7}} can become a side-wait protorun relatively easily. On the other hand, the {\LARGE\suo{2}\suo{2}\suo{4}} shape and the {\LARGE\suo{8}\suo{8}\suo{9}} shape are both weak; the first can become a side-wait protorun only if we draw {\LARGE\suo{5}}, and the second one will never become a side-wait protorun in one step. Therefore, both {\LARGE\suo{2}\suo{2}\suo{4}} and {\LARGE\suo{8}\suo{8}\suo{9}} should be kept as a candidate for the head or a group rather than making them into weak closed-wait protoruns.

\bigskip
Let's say we discard {\LARGE\wan{5}}, and then we draw {\LARGE\wan{8}}, resulting in the following hand. 
\begin{itembox}[r]{Advancing a hand 2}
\bp
\wan{5}\wan{7}\suo{2}\suo{2}\suo{4}\suo{8}\suo{8}\suo{9}\tong{2}\tong{3}\tong{7}\tong{8}\tong{9}~\wan{8}\\
\hfill\footnotesize{Draw~~~~~~~~~~~}
\ep
\vspace{-17pt}What would you discard? \vspace{-5pt}
\end{itembox}
\noindent
Now that we have a side-wait protorun {\LARGE\wan{7}\wan{8}}, we should discard {\LARGE\wan{5}}. 

\bigskip
Let's say we draw {\LARGE\suo{7}}, resulting in the following hand. 
\begin{itembox}[r]{Advancing a hand 3}
\bp
\wan{7}\wan{8}\suo{2}\suo{2}\suo{4}\suo{8}\suo{8}\suo{9}\tong{2}\tong{3}\tong{7}\tong{8}\tong{9}~\suo{7}\\
\hfill\footnotesize{Draw~~~~~~~~~~~}
\ep
\vspace{-17pt}What would you discard? \vspace{-5pt}
\end{itembox}
\noindent
This hand is now 1-away from ready, and our discard choice is between {\LARGE\suo{4}} and {\LARGE\suo{8}}. Both tiles are equally useless from our perspective, and so we will eventually discard them both. The question is which one we should discard first. Recall that a 4 is more versatile than an 8. This means that {\LARGE\suo{4}} in this hand may later become dangerous for the opponents; we should thus discard {\LARGE\suo{4}} now rather than later. 

\bigskip
Let's say we draw {\LARGE\tong{4}} after that, resulting in the following hand.
\begin{itembox}[r]{Advancing a hand 4}
\bp
\wan{7}\wan{8}\suo{2}\suo{2}\suo{7}\suo{8}\suo{8}\suo{9}\tong{2}\tong{3}\tong{7}\tong{8}\tong{9}~\tong{4}\\
\hfill\footnotesize{Draw~~~~~~~~~~~}
\ep
\vspace{-17pt}What would you discard? \vspace{-5pt}
\end{itembox}
\noindent
The hand is now ready. We should discard {\LARGE\suo{8}} and call riichi. If we win on {\LARGE\wan{9}}, we can claim riichi, {\jap pinfu}, and {\jap sanshoku} (Mixed Triple Chow), giving us 7700 points.\footnote{We will discuss scoring and {\jap yaku} more extensively in later chapters.}

\newpage
\section{Complex shapes}
The three basic types of tile blocks we have covered so far --- groups (runs and sets), protoruns (side wait, closed wait, and edge wait), and pairs --- form the basis of any standard mahjong hands.\footnote{Standard hands are those with four groups and one head. Non-standard hands are {\jap chiitoitsu} (Seven Pairs) and {\jap kokushi musou} (Thirteen Orphans).}  
When a hand has some tiles that do not constitute any of these three shapes, we treat them as {\bf floating tiles}. For example, {\LARGE\wan{5}} and {\LARGE\suo{6}} in the following hand are both floating tiles. 
\bigskip
\begin{itembox}[r]{Hand with floating tiles}
\bp
\wan{5}\wan{8}\wan{9}\tong{2}\tong{3}\tong{4}\tong{5}\tong{6}\tong{7}\suo{1}\suo{2}\suo{3}\suo{6}
\ep
\end{itembox}

\bigskip
In addition to these basic blocks, we often come across complex shapes that are made up of two or more groups, protoruns, pairs, and floating tiles combined. It is useful to comprehend such complex shapes as they are rather than breaking them up into smaller parts. We will discuss three-tile complex shapes and four-tile complex shapes in turn. 

\subsection{Three-tile complex shapes} 

There are two kinds of three-tile complex shapes --- double closed shape and protorun plus one shape.

\vfill
\subsubsection{Double closed ({\jap ryankan}) shape} \label{sec:ryankan}
	\index{double closed shape@double closed ({\jap ryankan}) shape} 
	\index{ryankan@{\jap ryankan} (double closed) shape}
When two closed-wait protoruns are combined, we have a {\bf double closed} ({\jap ryankan}) shape. 
There are five different patterns in each suit, as follows.
\bp
\suo{1}\suo{3}\suo{5}~~\suo{2}\suo{4}\suo{6}~~\suo{3}\suo{5}\suo{7}\\
\suo{4}\suo{6}\suo{8}~~\suo{5}\suo{7}\suo{9}
\ep
Each shape accepts as many as 2 kinds--8 tiles. For example, {\LARGE\suo{1}\suo{3}\suo{5}} accepts {\LARGE\suo{2}} (4 tiles) and {\LARGE\suo{4}} (4 tiles). This is twice as many as the number of tiles an isolated closed-wait protorun can accept. 

\bigskip
Sometimes a double closed shape is embedded within a tile block, making it difficult to detect it. 
For example, consider the following 1-away hand. 
\begin{itembox}[r]{Hand with a double closed shape}
\vspace{-5pt}
\bp
\hspace{-202pt}{\footnotesize\color{red!75!black} Red}\\ \vspace{-16pt}
\wan{4}\rfw\wan{5}\wan{7}\wan{8}\wan{9}\tong{4}\tong{5}\tong{6}\tong{6}\tong{7}\suo{7}\suo{7}~\tong{2}\\
\hfill\footnotesize{Draw~~~~~~~~~~~}
\ep
\vspace{-15pt}What would you discard? \vspace{-5pt}
\end{itembox}
\noindent
Before drawing {\LARGE\tong{2}}, the hand was already in a very good shape. It was perfect 1-away, accepting any of {\LARGE\wan{3}\wan{5}\wan{6}\tong{5}\tong{8}\suo{7}} (6 kinds--19 tiles). The question is whether we should keep {\LARGE\tong{2}} and discard {\LARGE\wan{5}} instead. \index{1-away (1-{\jap shanten})!perfect 1-away}

\bigskip
Notice that, if we keep {\LARGE\tong{2}}, we have a double closed shape {\LARGE\tong{2}\tong{4}\tong{6}}. This is because the block {\LARGE\tong{2}\tong{4}\tong{5}\tong{6}\tong{6}\tong{7}} can be split into {\LARGE\tong{2}\tong{4}\tong{6}} and {\LARGE\tong{5}\tong{6}\tong{7}}. 
If we keep {\LARGE\tong{2}} and discard {\LARGE\wan{5}}, the hand is still 1-away from ready, accepting {\LARGE\wan{3}\wan{6}\tong{3}\tong{5}\tong{8}} (5 kinds--19 tiles). The benefit of discarding {\LARGE\wan{5}} to keep the double closed shape is that the hand can \emph{always} be {\jap pinfu} when it is ready. On the other hand, discarding {\LARGE\tong{2}} means that the hand may become a {\jap yaku}-less hand when drawing {\LARGE\wan{5}} or {\LARGE\suo{7}}.

\bigskip
Double closed shapes are particularly useful when a hand is relatively far from ready (2-away or worse). As a hand advances, however, its usefulness diminishes because this block requires \underline{three} (not two) tiles even though it is not a complete group. 
Moreover, it will ultimately become a single closed-wait protorun when this block remains incomplete when the hand is ready. 
Therefore, we should not rely too much on a double closed shape. 
For example, consider the following two hands. 
\bp
\wan{1}\wan{1}\wan{3}\wan{4}\wan{5}\tong{6}\tong{7}\tong{8}\suo{2}\suo{4}\suo{6}\zhong\zhong~\wan{2}\\
\hfill\footnotesize{Draw~~~~~~~~~~~~~~~}
\ep 
\vspace{-20pt}
\bp
\wan{3}\wan{4}\wan{7}\wan{8}\wan{9}\tong{3}\tong{4}\tong{5}\tong{8}\tong{8}\suo{2}\suo{4}\suo{6}~\tong{7}\\
\hfill\footnotesize{Draw~~~~~~~~~~~~~~~}
\ep

Both hands are 1-away from ready and both contain a double closed shape in {\jap souzu} (bamboos) tiles. Maintaining the double closed shape in these cases will not be ideal. 
It is true that, if the hand becomes ready by drawing {\LARGE\suo{3}} or  {\LARGE\suo{5}} first, each of the hands makes for a good-wait ready hand. However, if the first hand becomes ready by calling {\jap pon} on {\LARGE\zhong} or the second hand becomes ready by drawing {\LARGE\wan{2}} or {\LARGE\wan{5}} first, they only make for a closed-wait ready hand. 

\bigskip
Therefore, when we draw a tile next to the head, creating a side-wait protorun, we should keep it and break the double closed shape instead. In the first example above, as we draw {\LARGE\wan{2}} that creates a side-wait protorun {\LARGE\wan{4}\wan{5}}, we should keep it and discard the {\LARGE\suo{2}} instead. In the second example above, as we draw {\LARGE\tong{7}} that creates a side-wait protorun {\LARGE\tong{7}\tong{8}}, we should keep it and discard {\LARGE\suo{2}} instead. 

\subsubsection{Protorun plus one shape}

As we saw with the first example in Section \ref{sec:ptat}, we often come across a tile combination such as {\LARGE\wan{5}\wan{5}\wan{7}} that is made up with one protorun plus one floating tile ({\LARGE\wan{5}\wan{7}} + {\LARGE\wan{5}}).\footnote{Alternatively, we can think of these combinations as a pair plus one {\wan{5}\wan{5} + \wan{7}}.}
Depending on the type of protoruns, we can classify protorun plus one shapes into three types, as summarized in Table \ref{tbl:protoone}. 

\bigskip

{\begin{table}[h!]\centering\small \captionsetup{font=footnotesize}
\caption{Types of protorun plus one shapes} \label{tbl:protoone}
\begin{tabular}{l c c c}
\toprule
Name & Example & Wait & Acceptance\\
\midrule
side wait +1 & {\LARGE \wan{3}\wan{3}\wan{4}} & {\LARGE \wan{2}-\wan{5} \wan{3}} & 3 kinds--10 tiles\\ [\sep]
closed wait +1 & {\LARGE \tong{2}\tong{2}\tong{4}} & {\LARGE \tong{2} \tong{3}} & 2 kinds--6 tiles\\ [\sep]
edge wait +1 & {\LARGE \suo{8}\suo{8}\suo{9}} & {\LARGE \suo{7} \suo{8}} & 2 kinds--6 tiles\\ [\sep]
\bottomrule
\end{tabular}
\end{table}}

A protorun plus one can accept two additional tiles that an isolated protorun cannot. This is because these blocks can now be a candidate for a set as well as for a run. 

\bigskip
Breaking a protorun plus one can be inefficient. For example, if we break a closed wait plus one shape into an isolated pair (i.e., discard {\LARGE\tong{4}} from {\LARGE \tong{2}\tong{2}\tong{4}}), the tile acceptance decreases from 6 to 2; it can accept only {\LARGE\tong{2}} (1 kind--2 tiles). Similarly, if we break it into an isolated protorun (i.e., discard {\LARGE\tong{2}} from {\LARGE \tong{2}\tong{2}\tong{4}}), the tile acceptance decreases from 6 to 4; it can accept only {\LARGE\tong{3}} (1 kind--4 tiles). With this in mind, consider the following hand.

\begin{itembox}[r]{Protorun plus one}
\bp
\wan{1}\wan{1}\wan{5}\wan{7}\tong{5}\suo{1}\suo{2}\suo{3}\suo{4}\suo{7}\suo{8}\suo{9}\suo{9}~\wan{5}\\
\hfill\footnotesize{Draw~~~~~~~~~~~}
\ep
\vspace{-15pt}What would you discard? \vspace{-5pt}
\end{itembox}
\noindent
Discarding {\LARGE\wan{5}} or {\LARGE\wan{7}} to break the protorun plus one {\LARGE\wan{5}\wan{5}\wan{7}} is inefficient here. Discarding {\LARGE\wan{5}} decreases tile acceptance by two, and discarding {\LARGE\wan{7}} decreases tile acceptance by four. Moreover, discarding {\LARGE\wan{7}} leaves three pairs in this hand, which should be avoided. Discarding {\LARGE\suo{9}} is much more efficient. 

\bigskip

Sometimes we have to make a choice between multiple protorun plus one shapes, just like we did in examples in Section \ref{sec:ptat}. 
Consider the following hand. What would you discard?
\begin{itembox}[r]{Multiple protoruns plus one}
\bp
\wan{4}\wan{4}\wan{6}\tong{3}\tong{3}\tong{4}\suo{1}\suo{2}\suo{3}\suo{4}\suo{6}\suo{8}\bei\bei
\ep
\vspace{-10pt}What would you discard? \vspace{-5pt}
\end{itembox}
\noindent
There are two protorun plus one shapes in this hand: {\LARGE\wan{4}\wan{4}\wan{6}} and {\LARGE\tong{3}\tong{3}\tong{4}}. We have to break one of the two into either an isolated Pair or an isolated protorun, because the other parts of this hand are more or less self-sufficient. Which one should we choose?

\bigskip
When choosing between which protoruns plus one to break, priority should be given to the weaker one. Since the side-wait protorun {\LARGE\tong{3}\tong{4}} is much stronger than the closed-wait protorun {\LARGE\wan{4}\wan{6}}, we should prioritize the latter and maintain {\LARGE\wan{4}\wan{4}\wan{6}}. In other words, the side-wait protorun {\LARGE\tong{3}\tong{4}} is so strong that we do not need to provide a cover by maintaining the ``plus one'' tile, {\LARGE\tong{3}}. On the other hand, the closed-wait protorun {\LARGE\wan{4}\wan{6}} is weaker so we should cover it by keeping another {\LARGE\wan{4}} as a back-up. You should thus discard {\LARGE\tong{3}}. 

\subsection{Four-tile complex shapes} 
Among several different kinds of four-tile complex shapes, we will focus on those that are made up of one group and one floating tile. There are three variants of this kind --- stretched single, bulging float, and skipping.

\subsubsection{Stretched single ({\jap nobetan}) shape} \label{sec:nobetan}
	\index{stretched single shape@stretched single ({\jap nobetan}) shape} 
	\index{nobetan@{\jap nobetan} (stretched single) shape}
A set of four consecutive tiles such as {\LARGE\tong{2}\tong{3}\tong{4}\tong{5}} is called a {\bf stretched single} ({\jap nobetan}) shape. Stretched single shapes are very useful both when a hand is ready and when a hand is 1-away or worse. 

\bigskip
When a stretched single shape is in a ready hand, that part forms the wait of the hand. For example, the following hand is ready, waiting for {\LARGE\wan{1} \wan{4}}.
\bigskip
\begin{itembox}[r]{Ready hand with a stretched single shape}
\bp
\wan{1}\wan{2}\wan{3}\wan{4}\tong{2}\tong{3}\tong{4}\suo{2}\suo{3}\suo{4}\zhong\zhong\zhong
\ep 
\end{itembox}
In a ready hand, the stretched single shape can be thought of as a candidate for the head ({\LARGE\wan{1}} or {\LARGE\wan{4}}) and a candidate for a run ({\LARGE\wan{2}\wan{3}\wan{4}} or {\LARGE\wan{1}\wan{2}\wan{3}}). For example, if we win this hand on {\LARGE\wan{1}}, then {\LARGE\wan{1}} becomes the head, and {\LARGE\wan{2}\wan{3}\wan{4}} becomes a run. On the other hand, if we win this hand on {\LARGE\wan{4}}, then {\LARGE\wan{4}} becomes the head, and {\LARGE\wan{1}\wan{2}\wan{3}} becomes a run. 

\bigskip

Another important role that a stretched single shape can play is to work as a candidate for two runs. When a hand is 1-away or worse, we can count on a stretched single shape to produce two runs. 
For example, consider a stretched single shape {\LARGE\tong{3}\tong{4}\tong{5}\tong{6}}. 
If we draw {\LARGE\tong{4}}, we will have a side-wait protorun {\LARGE\tong{3}\tong{4}} in addition to a complete run {\LARGE\tong{4}\tong{5}\tong{6}}. Similarly, if we draw {\LARGE\tong{5}}, we will have a side-wait protorun {\LARGE\tong{5}\tong{6}} in addition to a complete run {\LARGE\tong{3}\tong{4}\tong{5}}. Moreover, if we draw {\LARGE\tong{2}} or {\LARGE\tong{7}}, we will have a 3-way side-wait shape {\LARGE\tong{2}\tong{3}\tong{4}\tong{5}\tong{6}} (waiting for {\LARGE\tong{1}-\tong{4}-\tong{7}}) or {\LARGE\tong{3}\tong{4}\tong{5}\tong{6}\tong{7}} (waiting for {\LARGE\tong{2}-\tong{5}-\tong{8}}). 

\bigskip
There are six patterns of stretched single shapes, from 1234 through 6789. Table \ref{tbl:nobetan} summarizes the tiles each shape can accept to produce various waits. 

{\begin{table}[h!]\centering\small\captionsetup{font=footnotesize}
\caption{Types of stretched single shapes} \label{tbl:nobetan}
\begin{tabular}{l llll c}
\toprule
Shape & 3-way & 2-way & 1-way & Pair & Acceptance\\
\midrule
{\LARGE\wan{1}\wan{2}\wan{3}\wan{4}}
	&
	& {\LARGE \wan{3} \wan{5}}
	& {\LARGE \wan{2} \wan{6}}
	& {\LARGE \wan{1} \wan{4}}
	& 6 kinds--20 tiles\\ [\sep]
{\LARGE\wan{2}\wan{3}\wan{4}\wan{5}}
	& {\LARGE\wan{6}}
	& {\LARGE \wan{1} \wan{3} \wan{4}}
	& {\LARGE \wan{7}}
	& {\LARGE \wan{2} \wan{5}}
	& 7 kinds--24 tiles\\ [\sep]
{\LARGE\wan{3}\wan{4}\wan{5}\wan{6}}
	& {\LARGE\wan{2} \wan{7}}
	& {\LARGE \wan{4} \wan{5}}
	& {\LARGE \wan{1} \wan{8}}
	& {\LARGE \wan{3} \wan{6}}
	& 8 kinds--28 tiles\\ [\sep]
{\LARGE\wan{4}\wan{5}\wan{6}\wan{7}}
	& {\LARGE\wan{3} \wan{8}}
	& {\LARGE \wan{5} \wan{6}}
	& {\LARGE \wan{2} \wan{9}}
	& {\LARGE \wan{4} \wan{7}}
	& 8 kinds--28 tiles\\ [\sep]
{\LARGE\wan{5}\wan{6}\wan{7}\wan{8}}
	& {\LARGE\wan{4}}
	& {\LARGE \wan{6} \wan{7} \wan{9}}
	& {\LARGE \wan{3}}
	& {\LARGE \wan{5} \wan{8}}
	& 7 kinds--24 tiles\\ [\sep]
{\LARGE\wan{6}\wan{7}\wan{8}\wan{9}}
	&
	& {\LARGE \wan{5} \wan{7}}
	& {\LARGE \wan{4} \wan{8}}
	& {\LARGE \wan{6} \wan{9}}
	& 6 kinds--20 tiles\\ [\sep]
\bottomrule
\end{tabular}
\end{table}}

As we can see, the middle two ones --- 3456 and 4567 --- are the most versatile. They can accept two different tiles to produce a 3-way wait (27 or 38), two different tiles to produce a 2-way side wait (45 or 56), and two different tiles to produce a 1-way wait (18 or 29 to produce a closed wait). The 3456 and 4567 shapes are the most valuable of all four-tile shapes, and we should not lightly break such shapes when a hand is far away from ready. With this in mind, consider the following 2-away hand. 

\vfill
\bigskip
\begin{itembox}[r]{2-away hand with a stretched single shape}
\bp
\wan{3}\wan{4}\wan{5}\wan{6}\tong{1}\tong{1}\tong{3}\tong{5}\suo{1}\suo{2}\suo{6}\suo{6}\suo{7}\suo{9}
\ep
\vspace{-10pt}What would you discard? \vspace{-5pt}
\end{itembox}

\bigskip 
\noindent
It is true that discarding {\LARGE\wan{3}} or {\LARGE\wan{6}} would lead to the greatest tile acceptance (7 kinds--24 tiles) temporarily. However, doing so is too myopic. If we do that, all the remaining protoruns will be closed-wait or edge-wait ones. We should rather discard {\LARGE\suo{9}} to keep the 3456 shape, which we can expect to produce two side-wait protoruns later. The resulting tile acceptance (6 kinds--20 tiles) is not much smaller, either. 

\bigskip

\color{MyRed}
\begin{itembox}[c]{Four-tile complex shapes 1: {\jap nobetan}}
\normalcolor
Try to keep a stretched single shape if a hand has one. In particular, 3456 and 4567 should be kept until the hand becomes ready or 1-away from ready. 
\end{itembox}
\normalcolor

\bigskip


\subsubsection{Bulging float ({\jap nakabukure}) shape}
	\index{bulging@bulging float ({\jap nakabukure})} 
	\index{nakabukure@{\jap nakabukure} (bulging float)}

When we have a floating tile in the middle of a run (e.g., {\LARGE\wan{3}\wan{4}\wan{4}\wan{5}}), we have a {\bf bulging float} ({\jap nakabukure}) shape. 
Bulging float shapes are quite good at producing side-wait protoruns. Any bulging float shapes from 2334 through 6778 can accept four kinds of tiles to produce a side-wait protorun and a complete run. Take {\LARGE\wan{3}\wan{4}\wan{4}\wan{5}}, for example. It can produce a side-wait protorun and a complete run if we draw any of {\LARGE\wan{2}\wan{3}\wan{5}\wan{6}}. 
With this in mind, consider the following 2-away hand.
\bigskip
\begin{itembox}[r]{Hand with a bulging float shape}
\bp
\wan{3}\wan{4}\wan{4}\wan{5}\tong{4}\tong{6}\tong{8}\tong{8}\suo{1}\suo{3}\suo{5}\suo{5}\suo{6}\suo{8}
\ep
\vspace{-10pt}What would you discard? \vspace{-5pt}
\end{itembox}
Discarding {\LARGE\wan{4}} to break the bulging float shape is not ideal. Although doing so increases tile acceptance temporarily, the hand will be filled with closed-wait protoruns. Alternatively, you should discard {\LARGE\suo{8}} to maintain the bulging float shape. 

\bigskip
That being said, when this shape remains as is when a hand is ready, it does not make for a good wait. For example, consider the following ready hand. 
\begin{itembox}[r]{Ready hand with a bulging float shape}
\bp
\wan{3}\wan{4}\wan{4}\wan{5}\tong{2}\tong{3}\suo{2}\suo{3}\suo{4}\bei\zhong\zhong\zhong~\tong{4}\\
\hfill\footnotesize{Draw~~~~~~~~~~~}
\ep 
\vspace{-15pt}What would you discard? \vspace{-5pt}
\end{itembox}
\noindent
Discarding {\LARGE\bei} to keep the bulging float shape {\LARGE\wan{3}\wan{4}\wan{4}\wan{5}} makes the wait of this hand pretty bad. It is waiting for {\LARGE\wan{4}}, but we are already using two of it in the hand, leaving only two winning tiles. We should rather discard {\LARGE\wan{4}} to wait for {\LARGE\bei}. 

\vfill
\color{MyRed}
\begin{itembox}[c]{Four-tile complex shapes 2: {\jap nakabukure}}
\normalcolor
Try to keep a bulging float shape until a hand becomes 1-away.
\end{itembox}\normalcolor

\subsubsection{Skipping shape}
\index{skipping shape}
When we have a floating tile two tiles away from a run, we have a {\bf skipping shape}. For example, in a shape {\LARGE\wan{3}\wan{5}\wan{6}\wan{7}}, {\LARGE\wan{3}} is floating next next to a run {\LARGE\wan{5}\wan{6}\wan{7}}. {\LARGE\wan{3}} in a skipping shape is more valuable than isolated {\LARGE\wan{3}}, because it increases the kinds of tiles the hand can accept to produce a protorun or a 3-way side-wait shape. Table \ref{tbl:skipping} summarizes all the skipping shapes and the tiles each shape can accept. 

{\begin{table}[h!]\centering\small\captionsetup{font=footnotesize}
\caption{Types of skipping shapes} \label{tbl:skipping}
\begin{tabular}{l llll c}
\toprule
Shape & 3-way & 2-way & 1-way & Pair & Acceptance\\
\midrule
{\LARGE\wan{1}\wan{3}\wan{4}\wan{5}}
	&
	& {\LARGE \wan{2}}
	& {\LARGE \wan{3} \wan{6}}
	& {\LARGE \wan{1}}
	& 4 kinds--14 tiles\\ [\sep]
{\LARGE\wan{2}\wan{4}\wan{5}\wan{6}}
	& {\LARGE\wan{3}}
	& 
	& {\LARGE \wan{1} \wan{4} \wan{7}}
	& {\LARGE \wan{2}}
	& 5 kinds--18 tiles\\ [\sep]
{\LARGE\wan{3}\wan{5}\wan{6}\wan{7}}
	& {\LARGE\wan{4}}
	& {\LARGE\wan{2}}
	& {\LARGE \wan{1} \wan{5} \wan{8}}
	& {\LARGE \wan{3}}
	& 6 kinds--22 tiles\\ [\sep]
{\LARGE\wan{4}\wan{6}\wan{7}\wan{8}}
	& {\LARGE\wan{5}}
	& {\LARGE\wan{3}}
	& {\LARGE \wan{2} \wan{6} \wan{9}}
	& {\LARGE \wan{4}}
	& 6 kinds--22 tiles\\ [\sep]
{\LARGE\wan{5}\wan{7}\wan{8}\wan{9}}
	& 
	& {\LARGE\wan{4} \wan{6}}
	& {\LARGE\wan{3} \wan{7}}
	& {\LARGE \wan{5}}
	& 5 kinds--18 tiles\\ [\sep]
{\LARGE\wan{1}\wan{2}\wan{3}\wan{5}}
	& 
	& {\LARGE \wan{4} \wan{6}}
	& {\LARGE \wan{3} \wan{7}}
	& {\LARGE \wan{5}}
	& 5 kinds--18 tiles\\ [\sep]
{\LARGE\wan{2}\wan{3}\wan{4}\wan{6}}
	& {\LARGE\wan{5}}
	& {\LARGE\wan{7}}
	& {\LARGE \wan{1} \wan{4} \wan{8}}
	& {\LARGE \wan{6}}
	& 6 kinds--22 tiles\\ [\sep]
{\LARGE\wan{3}\wan{4}\wan{5}\wan{7}}
	& {\LARGE\wan{6}}
	& {\LARGE\wan{8}}
	& {\LARGE \wan{2} \wan{5} \wan{9}}
	& {\LARGE \wan{7}}
	& 6 kinds--22 tiles\\ [\sep]
{\LARGE\wan{4}\wan{5}\wan{6}\wan{8}}
	& {\LARGE\wan{7}}
	& 
	& {\LARGE \wan{3} \wan{6} \wan{9}}
	& {\LARGE \wan{8}}
	& 5 kinds--18 tiles\\ [\sep]
{\LARGE\wan{5}\wan{6}\wan{7}\wan{9}}
	&
	& {\LARGE \wan{8}}
	& {\LARGE \wan{4} \wan{7}}
	& {\LARGE \wan{9}}
	& 4 kinds--14 tiles\\ [\sep]
\bottomrule
\end{tabular}
\end{table}}

\bigskip
Bearing in mind that {\LARGE\wan{3}} of {\LARGE\wan{3}\wan{5}\wan{6}\wan{7}} is more valuable than isolated {\LARGE\wan{3}}, consider the following hand. 
\begin{itembox}[r]{Hand with a skipping shape}
\bp
\wan{3}\wan{7}\wan{8}\tong{5}\tong{6}\tong{7}\suo{2}\suo{4}\suo{6}\suo{7}\zhong\zhong\zhong~\tong{3}\\
\hfill\footnotesize{Draw~~~~~~~~~~~}
\ep
\vspace{-15pt}What would you discard? \vspace{-5pt}
\end{itembox}
\noindent
We should keep {\LARGE\tong{3}} and discard {\LARGE\wan{3}} instead. This is because {\LARGE\tong{3}} is a part of a skipping shape {\LARGE\tong{3}\tong{5}\tong{6}\tong{7}}, but {\LARGE\wan{3}} is an isolated floating tile. 

\bigskip
As we can see in Table \ref{tbl:skipping}, skipping shapes with a terminal tile (1345 and 5679) are also valuable. The 1 of 1345 and the 9 of 5679 can accept more tiles than an isolated 2 or 8 (let alone than an isolated 1 or 9). 

\newpage
\section{Waits} \label{sec:waits}
	\index{waits}
	\index{waits!side wait} \index{waits!closed wait} 
	\index{waits!dual pon wait} \index{waits!edge wait} \index{waits!single wait} 

There are five basic wait patterns, as summarized in Table \ref{tbl:waits}. More complicated wait patterns can emerge when some of these five basic patterns are combined. 

{\begin{table}[h!]\centering\small\captionsetup{font=footnotesize}
\caption{Five basic waits} \label{tbl:waits}
\begin{tabular}{l l c c c}
\toprule
Name & Japanese & Example & Wait & Acceptance\\
\midrule
side wait & {\jap ryanmen} & {\LARGE \wan{3}\wan{4}} & {\LARGE \wan{2}-\wan{5}} & 2 kinds--8 tiles\\ [\sep]
dual {\jap pon} wait & {\jap shanpon} & {\LARGE \suo{3}\suo{3}\tong{5}\tong{5}}& {\LARGE \suo{3} \tong{5}} & 2 kinds--4 tiles\\ [\sep]
closed wait & {\jap kanchan} & {\LARGE \suo{6}\suo{8}} & {\LARGE \suo{7}} & 1 kind--4 tiles\\ [\sep]
edge wait & {\jap penchan} & {\LARGE \tong{1}\tong{2}} & {\LARGE \tong{3}} & 1 kind--4 tiles\\ [\sep]
single wait & {\jap tanki} & {\LARGE \wan{2}} & {\LARGE \wan{2}} & 1 kind--3 tiles\\ [\sep]
\bottomrule
\end{tabular}
\end{table}}

\bigskip
As we can see in the table, side wait is the strongest of all the basic waits in terms of the kinds and the number of tiles to win on. 
Single wait appears to be much worse than others, but single-wait hands tend to have many possibilities of improving the wait and/or scores further. Moreover, single wait of an honor tile has a relatively high chance of winning it by {\jap ron}.

\subsubsection{Stretched single wait and semi side wait}
	\index{waits!stretched single wait} 
	\index{waits!semi side wait} 
Table \ref{tbl:waits2} summarizes two wait patterns, each of which can be thought of as a combination of some basic wait patterns. 
As I mentioned before, a stretched single shape in a ready hand forms a 2-way single wait. It is a decent wait pattern, as the number of tiles to win on (2 kinds--6 tiles) is twice as big compared with a regular single wait. 

{\begin{table}[h!]\centering\small\captionsetup{font=footnotesize}
\caption{Stretched single wait and semi side wait} \label{tbl:waits2}
\begin{tabular}{l c c c}
\toprule
Name & Example & Waits & Acceptance\\
\midrule
stretched single wait & {\LARGE \wan{2}\wan{3}\wan{4}\wan{5}} & {\LARGE \wan{2} \wan{5}} & 2 kinds--6 tiles\\ [\sep]
semi side wait & {\LARGE \suo{3}\suo{3}\suo{4}\suo{5}}& {\LARGE \suo{3}-\suo{6}} & 2 kinds--6 tiles\\ [\sep]
\bottomrule
\end{tabular}
\end{table}}

\bigskip
However, stretched single wait should not be confused with side wait for a few reasons. First, the number of tiles a 2-way stretched-single-wait hand can win on is at most 6, whereas it is 8 for a 2-way side-wait hand. The difference between 6 and 8 is non-trivial. Second, stretched single wait is still a variant of single wait, which means two things. On the one hand, we cannot claim {\jap pinfu} when the wait is stretched-single wait. For example, the following hand has no {\jap yaku} and hence we cannot win it by {\jap ron} without calling riichi.  
\bp
\wan{4}\wan{5}\wan{6}\tong{1}\tong{2}\tong{3}\tong{4}\suo{1}\suo{2}\suo{3}\suo{4}\suo{5}\suo{6}
\ep \index{fu@{\jap fu} (minipoint)} \index{minipoint ({\jap fu})}
On the other hand, we get 2 minipoints ({\jap fu}) with a stretched single wait. 
For example, if we win the following hand by drawing {\LARGE\tong{1}}, we get 40 minipoints (20 base minipoints + 8 for a concealed set of honor tiles + 2 for self-draw + 2 for single wait = 32, rounded up to 40).\footnote{We will discuss methods of scoring and minipoints calculations extensively in Chapter \ref{ch:scores}.}
\bp
\wan{4}\wan{5}\wan{6}\tong{1}\tong{2}\tong{3}\tong{4}\suo{1}\suo{2}\suo{3}\zhong\zhong\zhong
\ep

When we have a side-wait protorun right next to a pair (e.g., 1123, 2234, 7899, etc.), we call it semi side wait. We distinguish this from regular side wait for two reasons. First, the number of tiles to win on is smaller (6 rather than 8) because we are already using 2 of the 8 winning tiles in our hand. Second, we can treat this wait pattern either as single wait or as side wait, depending on which interpretation gives us a greater score. For example, consider the following hand. 
\bp
\wan{4}\wan{5}\wan{6}\tong{1}\tong{1}\tong{2}\tong{3}\suo{1}\suo{2}\suo{3}\suo{4}\suo{5}\suo{6}
\ep
We will treat the wait in this hand as side wait because that will give us {\jap pinfu}.
However, consider the following hand that has the exact same wait pattern: {\LARGE\tong{1}\tong{1}\tong{2}\tong{3}}. 
\bp
\wan{4}\wan{5}\wan{6}\tong{1}\tong{1}\tong{2}\tong{3}\suo{1}\suo{2}\suo{3}\zhong\zhong\zhong
\ep
If we win this hand by drawing {\LARGE\tong{1}}, we will treat the wait as single wait: {\LARGE\tong{1}} + {\LARGE\tong{1}\tong{2}\tong{3}}, which will give us 40 minipoints. If we treated the wait as side wait: {\LARGE\tong{1}\tong{1}} + {\LARGE\tong{2}\tong{3}}, we would get only 30 minipoints.
Of course, if we win this hand on {\LARGE\tong{4}}, we cannot think of the wait as single wait (because it is not). Similarly, if we win it by {\jap ron}, it does not make a difference if it is side wait or single wait (either way we get 40 minipoints). 

\newpage

\subsubsection{3-way side wait}
When a side-wait protorun is combined with an adjacent run, we get a regular 3-way side-wait pattern. There are only three of this kind, summarized in Table \ref{tbl:waits3}.

{\begin{table}[h!]\centering\captionsetup{font=footnotesize}\small
\caption{Regular 3-way side wait} \label{tbl:waits3}
\begin{tabular}{l c c}
\toprule
Example & Wait & Acceptance\\
\midrule
{\LARGE \wan{2}\wan{3}\wan{4}\wan{5}\wan{6}} & {\LARGE \wan{1}-\wan{4}-\wan{7}} & 3 kinds--11 tiles\\ [\sep]
{\LARGE \tong{3}\tong{4}\tong{5}\tong{6}\tong{7}} & {\LARGE \tong{2}-\tong{5}-\tong{8}} & 3 kinds--11 tiles\\ [\sep]
{\LARGE \suo{4}\suo{5}\suo{6}\suo{7}\suo{8}} & {\LARGE \suo{3}-\suo{6}-\suo{9}} & 3 kinds--11 tiles\\ [\sep]
\bottomrule
\end{tabular}
\end{table}}

\bigskip
When we have a stretched single shape or semi side-wait shape combined with an adjacent run, we also get a 3-way wait pattern. Table \ref{tbl:waits4} summarizes some examples. 

\bigskip
{\begin{table}[h!]\centering\captionsetup{font=footnotesize}\small
\caption{Some irregular 3-way waits} \label{tbl:waits4}
\begin{tabular}{l c c}
\toprule
Example & Wait & Acceptance\\
\midrule
{\LARGE \wan{1}\wan{2}\wan{3}\wan{4}\wan{5}\wan{6}\wan{7}} & {\LARGE \wan{1} \wan{4} \wan{7}} & 3 kinds--9 tiles\\ [\sep]
{\LARGE \tong{2}\tong{3}\tong{4}\tong{5}\tong{5}\tong{6}\tong{7}} & {\LARGE \tong{2} \tong{5}-\tong{8}} & 3 kinds--9 tiles\\ [\sep]
{\LARGE \suo{4}\suo{5}\suo{6}\suo{7}\suo{8}\suo{9}\suo{9}} & {\LARGE \suo{3}-\suo{6}-\suo{9}} & 3 kinds--9 tiles\\ [\sep]
\bottomrule
\end{tabular}
\end{table}}

\bigskip
Notice that the number of tiles to win on in each pattern is smaller than those for the regular 3-way side waits, although the kinds of tiles to win on are the same (either 1-4-7, 2-5-8, or 3-6-9). This is because we are already using some of the winning tiles within the hand. 

\bigskip
Notice also that not all the wait patterns qualify as side wait, so claiming {\jap pinfu} is not always possible (similarly, claiming single wait is not always possible). For example, the first pattern in Table \ref{tbl:waits4} is essentially a 3-way stretched single shape; none of the waits embedded in this shape qualifies as side wait. In the second pattern, if we win on {\LARGE\tong{2}}, the wait must be interpreted as single wait; if we win on {\LARGE\tong{8}}, the wait must be interpreted as side wait; and if we win on {\LARGE\tong{5}}, we adopt whichever interpretation that generates the higher score. In the third pattern, winning on {\LARGE\suo{9}} allows us to claim single wait if doing so gives us a higher score. 

\vfill
\subsubsection{Complex waits}
When a set is combined with a floating tile nearby, we get some complex wait patterns with multiple waits. Table \ref{tbl:waits5} summarizes a few examples of irregular waits that involve a set and a floating tile. 

{\begin{table}[h!]\centering \small\captionsetup{font=footnotesize}
\caption{Some irregular waits (set and a floating tile)} \label{tbl:waits5}
\begin{tabular}{l c c c}
\toprule
Example & Combination & Wait & Acceptance\\
\midrule
{\LARGE \wan{1}\wan{2}\wan{2}\wan{2}} & single and edge & {\LARGE \wan{1} \wan{3}} & 2 kinds--7 tiles\\ [\sep]
{\LARGE \tong{1}\tong{3}\tong{3}\tong{3}} & single and closed & {\LARGE \tong{1} \tong{2}} & 2 kinds--7 tiles\\ [\sep]
{\LARGE \suo{2}\suo{3}\suo{3}\suo{3}} & single and side & {\LARGE \suo{2} \suo{1}-\suo{4}} & 3 kinds--11 tiles\\ [\sep]
\bottomrule
\end{tabular}
\end{table}}


\bigskip
When a set is combined with a protorun, pair, or a four-tile shape, we get even more complicated waits. Table \ref{tbl:waits6} summarizes only a few representative examples. 

{\begin{table}[t!]\centering \small \captionsetup{font=footnotesize}
\caption{Some irregular waits (set and a protorun, pair, or a four-tile shape)} \begin{tabular}{l c c}
\toprule
Example & Wait & Acceptance\\
\midrule
{\LARGE \wan{1}\wan{1}\wan{2}\wan{2}\wan{2}\wan{3}\wan{3}} & {\LARGE \wan{1} \wan{2} \wan{3}} & 3 kinds--5 tiles\\ [\sep]
{\LARGE \tong{1}\tong{1}\tong{2}\tong{2}\tong{3}\tong{3}\tong{3}} & {\LARGE \tong{1} \tong{2} \tong{3}} & 3 kinds--5 tiles\\ [\sep]
{\LARGE \suo{1}\suo{1}\suo{2}\suo{2}\suo{3}\suo{3}\suo{4}\suo{4}\bei\bei} & {\LARGE \suo{1} \suo{4} \bei} & 3 kinds--6 tiles\\ [\sep]
{\LARGE \wan{5}\wan{5}\wan{5}\wan{6}\wan{7}\nan\nan} & {\LARGE \wan{5}-\wan{8} \nan} & 3 kinds--7 tiles\\ [\sep]
{\LARGE \tong{6}\tong{7}\tong{8}\tong{8}\tong{9}\tong{9}\tong{9}} & {\LARGE \tong{5}-\tong{8} \tong{7}} & 3 kinds--9 tiles\\ [\sep]
{\LARGE \suo{2}\suo{2}\suo{2}\suo{3}\suo{4}\suo{4}\suo{5}} & {\LARGE \suo{3}-\suo{6} \suo{4}} & 3 kinds--9 tiles\\ [\sep]
{\LARGE \wan{3}\wan{3}\wan{3}\wan{5}\wan{6}\wan{7}\wan{8}} & {\LARGE \wan{4} \wan{5} \wan{8}} & 3 kinds--10 tiles\\ [\sep]
{\LARGE \tong{1}\tong{1}\tong{1}\tong{3}\tong{5}\tong{5}\tong{5}} & {\LARGE \tong{2} \tong{3} \tong{4}} & 3 kinds--11 tiles\\ [\sep]
\bottomrule
\end{tabular}
\label{tbl:waits6}
\end{table}}


\clearpage

\section{Glossary}

\begin{description}
\item[Simple tiles ({\jap tanyao hai})] are tiles between 2 and 8.
\item[Terminal tiles ({\jap yaochu hai})] are 1 and 9.
\item[Honor tiles ({\jap jihai})] are non-number tiles (dragon tiles and wind tiles).
\item[Value tiles ({\jap fanpai} / {\jap yakuhai})] include dragon tiles, seat wind tiles, and prevailing wind tiles. We get one {\jap han} for a set of value tiles. 
\item[Valueless wind tiles ({\jap otakaze hai})] are wind tiles that are neither a prevailing wind tile nor a seat wind tile.
\item[Run (chow / sequence; {\jap shuntsu})] is a set of three consecutive number tiles.
\item[Set (pung / triplet; {\jap kotsu})] is a set of three identical tiles. 
\item[Quad (kong; {\jap kantsu})] is a set of four identical tiles. 
\item[Protorun ({\jap taatsu})] is a set of two tiles in the same suit that can become a run when one more tile is added. 
\item[Pair ({\jap toitsu})] is a set of two identical tiles. 
\item[Ready ({\jap tenpai})] is when a hand is ready to win. 
\item[1-away ({\jap 1-shanten})] is when a hand can be ready with one more tile. 
\item[Perfect 1-away] is when a 1-away hand has two side-wait protoruns and two pairs.
\item[Tile acceptance ({\jap ukeire})] refers to the kinds and the number of tiles a hand can accept.
\item[Stretched single ({\jap nobetan}) shape] is a set of four consecutive number tiles. 
\item[Bulging float ({\jap nakabukure}) shape] is a four-tile shape that is made up with a run and one floating tile in the middle of the run.
\item[Skipping shape] is a four-tile shape made up with a run and one floating tile located at two tiles away from the run.
\end{description}
